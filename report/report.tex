\documentclass{article}
\usepackage{fullpage}
\usepackage{multicol}
\usepackage[colorlinks=true,citecolor=black,linkcolor=black,urlcolor=black]{hyperref}

\title{
{\large 15-887: Planning, Execution, and Learning}\\
Project Report}
\author{Carlo Angiuli \and Ben Blum \and Michael Sullivan}
\date{December 3, 2012}

\begin{document}
\maketitle

\section{Introduction}

Program verification is the problem of showing that a computer program correctly
implements a specification of its intended behavior, or finding a
counterexample. These verifications can be either \emph{formal} or
\emph{informal}. In formal verification, one mathematically defines both the
specification and the program's own semantics, using techniques such as
substructural operational semantics~\cite{rob}, and proves that these coincide.
Informal verifications are prone to false positives or false negatives---but are
more immediately applicable to a broader range of programs. Informal analyses
are often used in systems research to reason about ill-specified code ``in the
real world'', which is especially useful for finding concurrency
errors~\cite{ben}.

{\em Planners} take a domain definition and initial state, and compute a
sequence of actions for achieving a goal; a planner is {\em complete} when its
failure to achieve a goal guarantees that no sequence of actions can achieve
that goal.
We investigate the correspondence between planning and program verification: the domain is the programming language (or the abstract machine), the initial state is the program being verified, the goal is the specification, and the plans are execution traces.
If a complete planner cannot plan for an invalid state, this constitutes a
formal verification that a program is safe (assuming that the planner is
correct).

We investigate how planners can be used for program verification in two particular domains:
first, for reasoning about functional and logic programs using an encoding
inspired by substructural operational semantics,
and second, for reasoning about concurrent computation with a focus on synchronisation algorithms.
Our results are mixed.
Of the (how many?) modern planners we used, none were able to efficiently execute lambda calculus programs. % TODO how many
On the other hand, we found planners were quite efficient at checking concurrent programs against their specifications,
and could even be used to reason about potentially nonterminating programs, which is a sticking point for many modern model checkers. % is this really true?
Inspired by this, we also provide a compiler for automatically generating PDDL
domains for concurrent programs from a more natural representation.
We discuss why modern planners proved inadequate for reasoning about functional programming, and also discuss the implications for concurrency verification.

\section{Functional and logic programming}

% Functional and logic programming

It has been 

\cite{Dixon}


Substructural operational semantics (SSOS) is a 

\subsection{Plus}



\section{Concurrent imperative programming}

\subsection{Threads Domain}

We encoded a simple assembly-like abstract machine with a fork/join primitive as a planning domain, which we call the ``threads'' domain.
Its predicates are used to indicate values of data and to indicate a program's instruction sequence.
Its actions represent the evaluation rules for the language's instruction set.

Some predicates express data:
\begin{itemize}
	\item \texttt{value ?name ?value} indicates the value of a variable with a given name.
	\item \texttt{succ ?n ?m} indicates the structure of integers. We were unable to find a planner that supported built-in integers, so for each problem we defined a sequence of literals (\texttt{n0}, \texttt{n1}, etc) and used the successor relation to abstractly express (very) simple arithmetic.
\end{itemize}

We now discuss the specifics of selected instructions. In the domain definition, each instruction has both a predicate (to be used in the representation of a program's instruction stream) and an action (which allows the planner to change the machine state when it encounters).

\begin{itemize}
	\item \texttt{set ?me ?next ?name ?value} - Assigns a value to a variable. \texttt{me} is the label for this instruction (all instructions share this pattern), \texttt{next} is the label of the next instruction to evaluate after this one (most instructions share this pattern). The increment, decrement, load, atomic-exchange, and atomic-add instructions are similar.
	\item \texttt{branch ?me ?name ?iftrue ?iffalse} - Flow control. Instead of \texttt{next}, there are two next instructions, selected between depending whether \texttt{name} is \texttt{n0}.
	\item \texttt{exit ?me} - Terminates execution of the current thread (or program).
	\item \texttt{fork ?me ?next ?child1 ?child2} - Runs \texttt{child1} and \texttt{child2} to completion (both must \texttt{exit}), and then advances to \texttt{next}.
\end{itemize}

Finally, a special predicate indicates the program counter (instruction pointer):
\begin{itemize}
	\item \texttt{eval ?instruction ?out} - The first argument is the label associated with the next instruction to be executed. Its second argument is the ``destination'' label, a special label used to identify which threads have terminated. There will be as many \texttt{eval} tokens as currently-running threads.
\end{itemize}

\subsection{Warmup: Data Races}

Our first test case for the threads domain was to demonstrate a simple data race between two threads. We show how two interleaving threads attempting to increment a shared variable can produce nondeterministically different results. The initial state of the planning problem expresses the following program (or various elaborations thereof, such as placing the accesses in a loop or adding a third thread):

	\begin{center}
	\begin{tabular}{ll}
	\multicolumn{2}{c}{\texttt{fork(thread1, thread2);}} \\
	& \\
	\texttt{thread1() \{} & \texttt{thread2() \{} \\
	\texttt{~~~~temp0 = x;\qquad} & \texttt{~~~~temp1 = x;} \\
	\texttt{~~~~temp0++;} & \texttt{~~~~temp1++;} \\
	\texttt{~~~~x = temp0;} & \texttt{~~~~x = temp1;} \\
	\texttt{\}} & \texttt{\}} \\
	\end{tabular}
	\end{center}

In this example, possible values for \texttt{x} at the end are 1 and 2. The problems with filenames starting in \texttt{prob1} demonstrate this problem; those with filenames starting in \texttt{prob2} and \texttt{prob3} are elaborations on it.

\subsection{Verifying Synchronisation Algorithms}

We next explored several different algorithms for {\em synchronisation} -- the problem of protecting designated ``critical sections'' of execution from unsafe concurrent access. Mutual exclusion algorithms are characterised by three properties: {\em mutual exclusion}, {\em bounded waiting}, and {\em progress}~\cite{de0u}. We next discuss how planners helped us verify these properties.

\begin{enumerate}
	\item {\bf Mutual Exclusion.} An algorithm that provides mutual exclusion does not allow multiple threads to be executing in the critical section simultaneously.
	To ensure that an algorithm provides mutual exclusion, we write programs of the following form, in which both threads modify a \texttt{num\_in\_section} counter to indicate when they're in the critical section:
	\begin{center}
	\begin{tabular}{ll}
	\multicolumn{2}{c}{\texttt{num\_in\_section = 0;}} \\
	\multicolumn{2}{c}{\texttt{fork(thread1, thread2);}} \\
	& \\
	\texttt{thread1() \{} & \texttt{thread2() \{} \\
	\texttt{~~~~while (true) \{} & \texttt{~~~~while (true) \{} \\
	\texttt{~~~~~~~~lock\_sequence();} & \texttt{~~~~~~~~lock\_sequence();} \\
	\texttt{~~~~~~~~num\_in\_section++;} & \texttt{~~~~~~~~num\_in\_section++;} \\
	\texttt{~~~~~~~~num\_in\_section--;} & \texttt{~~~~~~~~num\_in\_section--;} \\
	\texttt{~~~~~~~~unlock\_sequence();\qquad} & \texttt{~~~~~~~~unlock\_sequence();} \\
	\texttt{~~~~\}} & \texttt{~~~~\}} \\
	\texttt{\}} & \texttt{\}} \\
	\end{tabular}
	\end{center}
	We then set the goal statement to be \texttt{num\_in\_section == 2}. When a complete planner cannot plan for this fact, it guarantees that no execution interleaving exists in which both threads are in the section at once -- in other words, that mutual exclusion is satisfied.
	\item {\bf Bounded Waiting.} An algorithm that provides  
	\item {\bf Progress.} An algorithm that provides
\end{enumerate}

\subsection{Compiler}
Sully - concurrent compiler


\section{Conclusion and Future Work}

In this project, we explored the correspondence between planning and program verification. Planning domains can be used to express an abstract machine's operational semantics, on top of which programs to be verified can be encoded as planning problems. Goal statements represent a program's specification, and output plans are execution traces.

When applied to functional programming idioms such as the SKI combinator calculus, we found planners were unable to verify any but the simplest programs due to memory requirements during evaluation. We found imperative abstract machines more workable, and verified several synchronisation algorithms.

This approach to formal concurrency verification, whether performed with planners or more conventional model checkers, may be applicable in the growing world of multiprocessor concurrency. Architectures such as ARM (common in cell phones) with weakly-consistent memory models are becoming more popular, and provide opportunities for new, difficult to reason about concurrency routines. Encoding multiprocessor memory semantics as abstract machines would be a logical progression of the work we presented here.

\vspace{1em}
Our codebase, comprising the functional and imperative domains and problems, and the threads compiler, can be viewed at \url{https://github.com/cangiuli/planning}.

\bibliography{citations}{}
\bibliographystyle{alpha}

\newpage

\appendix
\section{Code examples}
\label{sec:appendix}

Here we demonstrate the input format (C-like) and output format (PDDL) of our threads compiler with an example implementation of Dekker's algorithm.

\begin{figure}[h]
\begin{center}
\begin{verbatim}
int flag0 = 0; int flag1 = 0; int turn = 0;

int num_in_section = 0;
int thread1_waiting = 0; int thread2_iters = 0;

thread0() {
    while (1) {
        flag0 = 1;
        thread2_iters = 0; thread1_waiting = 1;

        while (flag1) {
            if (turn) { /* turn != 0 */
                flag0 = 0;
                while (turn) { /* turn != 0 */
                    /* busy wait */
                }
                flag0 = 1;
            }
        }

        /* critical section */
        thread1_waiting = 0;
        num_in_section++;
        num_in_section--;

        turn = 1;
        flag0 = 0;
    }
}

thread1() { /* ELIDED */ }

main() {
    fork(thread0, thread1);
}
\end{verbatim}
\end{center}
\caption{Dekker's algorithm in our simple language}
\label{fig:dekker-code}
\end{figure}

\begin{figure}[h]
\begin{center}
\small
\begin{verbatim}
(define (problem dekker-loop)
    (:domain threads)
    (:objects
        n0 n1 n2 n3 n4 n5 n6 - number
        out - label
        flag0 flag1 turn num_in_section thread1_waiting thread2_iters - label

        thread00 thread01 thread02 thread03 thread04 thread05 thread06 thread07
        thread08 thread09 thread010 thread011 thread012 thread013
        thread10 thread11 thread12 thread13 thread14 thread15 thread16 thread17
        thread18 thread19 thread110 thread111
        main0 main1
        - label
    )
    (:init
        (succ n0 n1) (succ n1 n2) (succ n2 n3)
        (succ n3 n4) (succ n4 n5) (succ n5 n6)
        ; .data
        (value flag0 n0)
        (value flag1 n0)
        (value turn n0)
        (value num_in_section n0)
        (value thread1_waiting n0)
        (value thread2_iters n0)

        ; .text
        ; thread0
        (set thread00 thread01 flag0 n1)
        (set thread01 thread02 thread2_iters n0)
        (set thread02 thread03 thread1_waiting n1)
        (branch thread03 flag1 thread04 thread08)
        (branch thread04 turn thread05 thread03)
        (set thread05 thread06 flag0 n0)
        (branch thread06 turn thread06 thread07)
        (set thread07 thread03 flag0 n1)
        (set thread08 thread09 thread1_waiting n0)
        (incr thread09 thread010 num_in_section)
        (decr thread010 thread011 num_in_section)
        (set thread011 thread012 turn n1)
        (set thread012 thread00 flag0 n0)
        (exit thread013)

        ; thread1
        ; ELIDED
        ; main
        (fork main0 main1 thread00 thread10)
        (exit main1)

        (eval main0 out)
    )
    (:goal (and
            ; INSERT GOALS HERE
        )
    )
)
\end{verbatim}
\end{center}
\caption{The corresponding generated problem for Dekker's algorithm}
\label{fig:dekker-asm}
\end{figure}



\end{document}

The report should be 4-6 pages and include descriptions of (1) the problem,
(2) your approach, and (3) your results.  Please highlight the planning aspects
and other course concepts in all of the sections.  You may want to include some
references to related work in the background for your problem description or in
the evaluation of your results, but you do not need to include a separate
related work survey.

