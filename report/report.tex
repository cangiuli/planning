\documentclass{article}
\usepackage{fullpage}
\usepackage{multicol}
\usepackage[colorlinks=true,citecolor=black,linkcolor=black,urlcolor=black]{hyperref}

\title{
{\large 15-887: Planning, Execution, and Learning}\\
Project Report}
\author{Carlo Angiuli \and Ben Blum \and Michael Sullivan}
\date{December 3, 2012}

\begin{document}
\maketitle

\section{Introduction}

Program verification is the problem of showing that a computer program correctly
implements a specification of its intended behavior, or finding a
counterexample. These verifications can be either \emph{formal} or
\emph{informal}. In formal verification, one mathematically defines both the
specification and the program's own semantics, using techniques such as
substructural operational semantics~\cite{rob}, and proves that these coincide.

Informal verifications are prone to false positives or false negatives---but are
more immediately applicable to a broader range of programs. Informal analyses
are often used in systems research to reason about ill-specified code ``in the
real world'', which is especially useful for finding concurrency
errors~\cite{ben}.

{\em Planners} take a domain definition and initial state, and compute a
sequence of actions for achieving a goal; a planner is {\em complete} when its
failure to achieve a goal guarantees that no sequence of actions can achieve
that goal.
We investigate the correspondence between planning and program verification: the domain is the programming language (or the abstract machine), the initial state is the program being verified, the goal is the specification, and the plans are execution traces.
If a complete planner cannot plan for an invalid state, this constitutes a
formal verification that a program is safe (assuming that the planner is
correct).

We investigate how planners can be used for program verification in two particular domains:
first, for reasoning about functional and logic programs using an encoding
inspired by substructural operational semantics,
and second, for reasoning about concurrent computation with a focus on synchronisation algorithms.
Our results are mixed.
Of the (how many?) modern planners we used, none were able to efficiently execute lambda calculus programs. % TODO how many
On the other hand, we found planners were quite efficient at checking concurrent programs against their specifications,
and could even be used to reason about potentially nonterminating programs, which is a sticking point for many modern model checkers. % is this really true?
Inspired by this, we also provide a compiler for automatically generating PDDL
domains for concurrent programs from a more natural representation.
We discuss why modern planners proved inadequate for reasoning about functional programming, and also discuss the implications for concurrency verification.

\section{Functional and logic programming}

% Functional and logic programming

It has been 

\cite{Dixon}


Substructural operational semantics (SSOS) is a 

\subsection{Plus}



\section{Concurrent imperative programming}

\subsection{Threads Domain}

We encoded a simple assembly-like abstract machine with a fork/join primitive as a planning domain, which we call the ``threads'' domain.
Its predicates are used to indicate values of data and to indicate a program's instruction sequence.
Its actions represent the evaluation rules for the language's instruction set.

Some predicates express data:
\begin{itemize}
	\item \texttt{value ?name ?value} indicates the value of a variable with a given name.
	\item \texttt{succ ?n ?m} indicates the structure of integers. We were unable to find a planner that supported built-in integers, so for each problem we defined a sequence of literals (\texttt{n0}, \texttt{n1}, etc) and used the successor relation to abstractly express (very) simple arithmetic.
\end{itemize}

We now discuss the specifics of selected instructions. In the domain definition, each instruction has both a predicate (to be used in the representation of a program's instruction stream) and an action (which allows the planner to change the machine state when it encounters).

\begin{itemize}
	\item \texttt{set ?me ?next ?name ?value} - Assigns a value to a variable. \texttt{me} is the label for this instruction (all instructions share this pattern), \texttt{next} is the label of the next instruction to evaluate after this one (most instructions share this pattern). The increment, decrement, load, atomic-exchange, and atomic-add instructions are similar.
	\item \texttt{branch ?me ?name ?iftrue ?iffalse} - Flow control. Instead of \texttt{next}, there are two next instructions, selected between depending whether \texttt{name} is \texttt{n0}.
	\item \texttt{exit ?me} - Terminates execution of the current thread (or program).
	\item \texttt{fork ?me ?next ?child1 ?child2} - Runs \texttt{child1} and \texttt{child2} to completion (both must \texttt{exit}), and then advances to \texttt{next}.
\end{itemize}

Finally, a special predicate indicates the program counter (instruction pointer):
\begin{itemize}
	\item \texttt{eval ?instruction ?out} - The first argument is the label associated with the next instruction to be executed. Its second argument is the ``destination'' label, a special label used to identify which threads have terminated. There will be as many \texttt{eval} tokens as currently-running threads.
\end{itemize}

\subsection{Warmup: Data Races}

Our first test case for the threads domain was to demonstrate a simple data race between two threads. We show how two interleaving threads attempting to increment a shared variable can produce nondeterministically different results. The initial state of the planning problem expresses the following program (or various elaborations thereof, such as placing the accesses in a loop or adding a third thread):

	\begin{center}
	\begin{tabular}{ll}
	\multicolumn{2}{c}{\texttt{fork(thread1, thread2);}} \\
	& \\
	\texttt{thread1() \{} & \texttt{thread2() \{} \\
	\texttt{~~~~temp0 = x;\qquad} & \texttt{~~~~temp1 = x;} \\
	\texttt{~~~~temp0++;} & \texttt{~~~~temp1++;} \\
	\texttt{~~~~x = temp0;} & \texttt{~~~~x = temp1;} \\
	\texttt{\}} & \texttt{\}} \\
	\end{tabular}
	\end{center}

In this example, possible values for \texttt{x} at the end are 1 and 2. The problems with filenames starting in \texttt{prob1} demonstrate this problem; those with filenames starting in \texttt{prob2} and \texttt{prob3} are elaborations on it.

\subsection{Verifying Synchronisation Algorithms}

We next explored several different algorithms for {\em synchronisation} -- the problem of protecting designated ``critical sections'' of execution from unsafe concurrent access. Mutual exclusion algorithms are characterised by three properties: {\em mutual exclusion}, {\em bounded waiting}, and {\em progress}~\cite{de0u}. We next discuss how planners helped us verify these properties.

\begin{enumerate}
	\item {\bf Mutual Exclusion.} An algorithm that provides mutual exclusion does not allow multiple threads to be executing in the critical section simultaneously.
	To ensure that an algorithm provides mutual exclusion, we write programs of the following form, in which both threads modify a \texttt{num\_in\_section} counter to indicate when they're in the critical section:
	\begin{center}
	\begin{tabular}{ll}
	\multicolumn{2}{c}{\texttt{num\_in\_section = 0;}} \\
	\multicolumn{2}{c}{\texttt{fork(thread1, thread2);}} \\
	& \\
	\texttt{thread1() \{} & \texttt{thread2() \{} \\
	\texttt{~~~~while (true) \{} & \texttt{~~~~while (true) \{} \\
	\texttt{~~~~~~~~lock\_sequence();} & \texttt{~~~~~~~~lock\_sequence();} \\
	\texttt{~~~~~~~~num\_in\_section++;} & \texttt{~~~~~~~~num\_in\_section++;} \\
	\texttt{~~~~~~~~num\_in\_section--;} & \texttt{~~~~~~~~num\_in\_section--;} \\
	\texttt{~~~~~~~~unlock\_sequence();\qquad} & \texttt{~~~~~~~~unlock\_sequence();} \\
	\texttt{~~~~\}} & \texttt{~~~~\}} \\
	\texttt{\}} & \texttt{\}} \\
	\end{tabular}
	\end{center}
	We then set the goal statement to be \texttt{num\_in\_section == 2}. When a complete planner cannot plan for this fact, it guarantees that no execution interleaving exists in which both threads are in the section at once -- in other words, that mutual exclusion is satisfied.
	\item {\bf Bounded Waiting.} An algorithm that provides  
	\item {\bf Progress.} An algorithm that provides
\end{enumerate}

\subsection{Compiler}
Sully - concurrent compiler


\bibliography{citations}{}
\bibliographystyle{alpha}

\end{document}

The report should be 4-6 pages and include descriptions of (1) the problem,
(2) your approach, and (3) your results.  Please highlight the planning aspects
and other course concepts in all of the sections.  You may want to include some
references to related work in the background for your problem description or in
the evaluation of your results, but you do not need to include a separate
related work survey.

